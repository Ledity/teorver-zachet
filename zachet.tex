\documentclass[a4paper,11pt,oneside]{article}

\usepackage{amsmath,amsthm,amssymb,cancel}
\usepackage[T1,T2A]{fontenc}
\usepackage[utf8]{inputenc}
\usepackage[english,russian]{babel}

% \usepackage{xcolor}
\usepackage{hyperref}

\newcommand*\rfrac[2]{{}^{#1}\!/_{#2}}

\title{Ответы на зачет по теории вероятностей}
\author{Воейко Андрей}
\date{2022}

\begin{document}
\maketitle

\section{Классическое и статистическое определение вероятности события.}
\textbf{Определение.} Классическое определение вероятности события:
\[
  P(A) = \frac{m}{n},
\]
где $P(A)$ --- вероятность события $A$, $n$ --- число испытаний, $m$ --- число благоприятных исходов.\newline
\textbf{Определение.} Статиситическое определение вероятности события:
\[
  P(A) = \lim{n \to \infty} \frac{m}{n},
\]
где $P(A)$ --- вероятность события $A$, $n$ --- число испытаний, $m$ --- число благоприятных исходов.
\section{Теорема сложения вероятностей несовметсных событий.}
\textbf{Теорема.} Пусть веротность события $A$ равна $P(A)$, а события $B$ --- $P(B)$, причем события $A$ и $B$ несовметсны. Тогда вероятность суммы двух этих событий $P(A + B)$ равна $P(A) + P(B)$.\newline
\textbf{Доказательство.} Пусть было произведено $n$ испытаний, из которых $m_{A}$ благоприятствовали $A$, а $m_{B}$ --- $B$. Тогда
\[
  P(A + B) = \frac{m_{A} + m_{B}}{n} = \frac{m_{A}}{n} + \frac{m_{B}}{n} = P(A) + P(B).
\]
Ч.Т.Д.
\section{Независимость событий. Условная вероятность.}
\subsection{Независимость событий.}
\textbf{Определение.} Независимыми называются такие два события, наступление одного из которых не влияет на вероятность наступления другого.
\subsection{Условная вероятность.}
\textbf{Определение.} Условной вероятностью называют вероятность наступления события $B$ при условии наступления события $A$.\newline
Обозначают:
\[
  P_{A}(B) \quad \text{или} \quad P(A|B).
\]
При этом
\[
  P_{A}(B) = \frac{m_{AB}}{m_{A}} = \frac{\rfrac{m_{AB}}{n}}{\rfrac{m_{A}}{n}} = \frac{P(AB)}{P(B)}.
\]
\section{Теорема умножения вероятностей}
\textbf{Теорема.} Вероятность произведения событий $A$ и $B$ равна произведению вероятностей $P(A) \cdot P_{A}(B)$.\newline
\textbf{Доказательство.}
\[
  P(A) \cdot P_{A}(B) = \frac{\cancel{m_{A}}}{n} \cdot \frac{m_{AB}}{\cancel{m_{A}}} = \frac{m_{AB}}{n}.
\]
Если события независимы, то $P_{A}(B) = P(B)$. Тогда
\[
  P(AB) = P(A) \cdot P_{A}(B) = P(A) \cdot P(B).
\]
\section{Теорема сложения вероятностей совместных событий.}
\textbf{Теорема.} Если два события $A$ и $B$ совместны, то вероятность суммы этих событий $P(A + B)$ равна $P(A) + P(B) - P(AB)$.
\textbf{Доказательство.}
\[
  P(A) + P(B) - P(AB) =
  \frac{m_{A}}{n} + \frac{m_{B}}{n} - \frac{m_{AB}}{n} =
  \frac{m_{A} + m_{B} - m_{AB}}{n}.
\]
Поскольку $AB \subset A$ и $AB \subset B$, при сложении $m_{A} + m_{B}$ мы учтем $m_{AB}$ два раза. Поэтому, $m_{A} + m_{B} - m_{AB}$ --- это и есть искомое нами количество благоприятных для $A + B$ исходов. Поэтому $P(A+B) = \frac{m_{A} + m_{B} - m_{AB}}{n} = P(A) + P(B) - P(AB)$.
\end{document}
